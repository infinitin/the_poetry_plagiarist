%%% Thesis Introduction --------------------------------------------------
\chapter{Introduction}
\ifpdf
    \graphicspath{{Introduction/IntroductionFigs/PNG/}{Introduction/IntroductionFigs/PDF/}{Introduction/IntroductionFigs/}}
\else
    \graphicspath{{Introduction/IntroductionFigs/EPS/}{Introduction/IntroductionFigs/}}
\fi

\section{Motivation}
Alan Turing devised the Turing Test as a standard for a computer's ability to exhibit behaviour indistinguishable to that of a human. The original example that he used to indicate the requirements for passing the test included the following:

\textit{\textbf{Q:} Please write me a sonnet on the subject of the Forth Bridge.}\\
\textit{\textbf{A:} Count me out on this one. I never could write poetry.}

Poetry is a linguistic art form whose works convey a particular message, often using symbolism and rhetoric. Writing poetry requires precise control over syntax, in-depth understanding of semantics and the contextual awareness to retain a coherent topic flow. They are often read out loud and use a number of mnemonic techniques based on the sound of speech, such as rhyme and rhythm.  Poets break grammatical rules under a \textit{"poetic license"} if it is believed that it will help their message resonate with the listener.

It's not unreasonable for Turing to have been pessimistic about the possibility of computers writing poetry. However, recent advances in Computational Linguistics and Creativity have made it worth investigation. In this paper, we attempt to create a system that can automatically generate poetry that makes sense grammatically, semantically and with some level of coherence.

Many types of poetry have spawned and evolved over millennia, each defined by the way it uses the aforementioned poetic techniques. We will not be able to write a variety of poems if we hard-code each style. Therefore, we aim that this system is able to learn the structure and typical content of classes of poems from examples and use this information to guide the generation process.

Poems are fundamentally \textit{creative} works, and creativity is a behaviour exhibited particularly well by humans. For computer-generated poems to truly be indistinguishable from those created by humans, we want to demonstrate some element of imagination and originality. To investigate this, certain philosophies of creativity have been broken down into logical components and implemented in code by researchers of Computational Creativity. It will be fascinating to see how they perform in practice through creation of poetry.

Ultimately, poems are created to be enjoyed and to help people express themselves, while the purpose of computers is primarily to aid people in as many ways as possible. We intend that our system can be used to assist people either by creating poems for them, or as a tool for those who wish to write their own.

%-----------------------------------

\section{Objectives}
We wish to create a system that:
\begin{itemize}
\item{identifies the usage of a broad list of poetic and linguistic features in a single poem (Chapter \ref{chp:analysis}).}
\item{learns usage patterns of these features and typical content of any class of poems from examples (Chapter \ref{chp:interpretation}).}
\item{produces poems (Chapter \ref{chp:generation}), spontaneously or using 'seeds' of inspiration, that are:}
	\begin{itemize}
	\setlength{\itemsep}{0pt}
	\item{novel,}
	\item{syntactically valid,}
	\item{semantically interpretable,}
	\item{contextually coherent,}
	\item{evident of desire poetic techniques,}
	\item{indicative of creativity and imagination.}
	\end{itemize} 
\item{can be used to assist people in the creation of poems.}
\end{itemize}

%---------------------------------------------------

\section{Contributions}
This project makes contributions towards both Computational Creativity and Computational Linguistics.
\begin{itemize}
\item{We demonstrate the ability for computer systems to assess written natural language works in terms of its structure, common words and phrases, rhetoric and poetic features such as rhyme, rhythm and alliteration.}
\item{We validate the commonly taught expectations on structure and content of popular types of poetry.}
\item{We will investigate the appropriateness of semantic networks of common sense as a representation of a written text's context in terms of objects and their descriptions, actions and relationships.}
\item {We demonstrate the ability to learn usage patterns of written features from the analysis of a large number of comparable texts.}
\item{We will investigate the appropriateness of semantic networks of common sense as a source of knowledge from which we can build sentences, retain a cohesive theme and demonstrate creativity.}
\item{We explore the effectiveness of a content-first approach to poetry and applicability of third-party resources for Natural Language Generation.}
\end{itemize}

%%% ----------------------------------------------------------------------

%%% Local Variables: 
%%% mode: latex
%%% TeX-master: "../thesis"
%%% End: 
