\chapter{Evaluation Plan}
\ifpdf
    \graphicspath{{Evaluation/EvaluationFigs/PNG/}{Evaluation/EvaluationFigs/PDF/}{Evaluation/EvaluationFigs/}}
\else
    \graphicspath{{Evaluation/EvaluationFigs/EPS/}{Evaluation/EvaluationFigs/}}
\fi

We will employ four methods of evaluating this project.

\section{Comparison of Analysis  Results to Theory}
The Analysis and Abstraction phases attempt to discover the common features of any type of poetry. Current theory of these common features have been derived and documented by poets through their own analysis. We wish to investigate whether this system is able to find everything documented that has been documented by poets. It is likely that the system could only find a subset but it will be interesting to see which points were missed. A full comparison will be made and explanation for any missing features will be attempted, as well as possible algorithms to detect them that could have been implemented. 
\\\\
A particularly interesting result would be if the system manages to find a common feature of a type of poetry that poets have not yet found. If even one case of this occurs then this would make a strong case for the ability of computers to understand, interpret and find patterns in natural language. Further investigation would surely be warranted into how much a computer system would be able to find that human readers have overlooked.
\\\\
Note that this section does not expect the system to infer the effects of any of the poetry features on the reader, just simply detect the use of the features.

\section{Turing-style Tests}
The simplest way to check the quality of poems produced would be to show people a poem either generated by this system or written by a human without telling them and asking them to determine whether the author was man or machine.
\\\\
However, this is highly dependent on the reader's fluency of English, understanding of poetry and imagination. A poor English speaker with little to no understanding of poetry and with a far reaching imagination could easily be fooled into thinking that a piece of text was written by a human rather than a computer.
\\\\
Therefore, we propose a survey that asks for these three measures to be told truthfully. It then randomly shows a poem and asks whether it was written by a human or computer. This way, we can get a demographic of those who are fooled and those who are not so that we can see at what level of poetry literacy this system is.
\\\\
Further, we plan to approach real poets from literary institutions and university departments and ask them to do this survey in person. If they incorrectly believe even one poem to be written by a human and not a computer, then this project is a definite success. For the cases where this does not arise, having an in-person survey will enable us to ask for feedback on what gave us away, which can be used to improve the system and future attempts at poetry generation.
\\\\
However, Pease and Colton 2011b make several arguments that Turing-style tests are not appropriate for poetry generation. We believe that it has its place because ultimately this project is for user consumption as well as research and experimentation. Furthermore, poems attempt to create an emotional connection with the reader, something that cannot be determined other than with a human reader.
\\\\
Having said that, Colton, Charnley and Pease 2011 described the FACE and IDEA Descriptive Models for evaluating computational creativity projects. We believe these models provide a useful evaluation methodology alongside Turing-style tests and so are just as much part of the evaluation of this system as described in Section 4.3 and 4.4.

\section{FACE Descriptive Model}
A full FACE model has four symptoms: examples, concepts, aesthetics and framing information.

\begin{itemize}
\setlength{\itemsep}{0pt}
\item{\emph{Examples} will be showcased by the templates generated by the Analysis and Abstraction phase.}
\item{\emph{Concepts} are of the form of the algorithm described for the Generation phase as it takes input from the user or online, the results from the Abstraction phase and several third party libraries to output a poem.}
\item{\emph{Aesthetics} are assessed by running the Analysis phase over the poem again. In fact, this happens several times during the creation of the poem. Any faults are reported back to the next iteration of that poem.}
\item{\emph{Framing Information} is the poem created.}
\end{itemize}

Therefore, we can see that this system should fully abide by the FACE Descriptive Model. We will evaluate the results mathematically as per Colton, Charnley and Pease 2011.

\section{IDEA Descriptive Model}
A full IDEA model has six stages to which the software can reach. We want our software to be in the, fourth or \emph{Discovery stage}.

\begin{itemize}
\setlength{\itemsep}{0pt}
\item{\emph{Developmental stage}: this system has a full Abstraction phase to avoid the case that all creative acts undertaken by this system are purely based on inspiring examples. So this system will have surpassed this stage.}
\item{\emph{Fine-tuning stage}: the Abstraction phase only looks for a limited number of overlapping features to provide the template, leading to higher level abstraction. For example, it does not use part-of-speech tags from previous examples or any low level abstractions. We believe the system should be able to surpass this level.}
\item{\emph{Re-invention stage}: the system is able to work off a template provided by the Abstraction phase, but also able to mutate the templates and add or remove restrictions both automatically and guided by the user. Therefore, the creative acts are not restricted only to those that are known and should be able to surpass this stage as well.}
\item{\emph{Discovery stage}: the ability to work off templates derived from Analysis and Abstraction imply that the system is able to generate works that are sufficiently similar to be assessed with current contexts. However, given the flexibility of the mutation and user-guidance ability, it can also produce works that are significantly dissimilar. We believe the system to be able to reach this stage.}
\item{\emph{Disruption and Disorientation stages}: Since templates and constraints on the creative work that are imposed are the results of analysing and abstracting existing works, it is not the case that this system solely produces poetry that is too dissimilar to those known by theory. }
\end{itemize}

Therefore, we can see that this system should reach the desired \emph{Discovery stage} of the IDEA Descriptive Model. We will evaluate the results mathematically as per Colton, Charnley and Pease 2011.

% ------------------------------------------------------------------------


%%% Local Variables: 
%%% mode: latex
%%% TeX-master: "../thesis"
%%% End: 
