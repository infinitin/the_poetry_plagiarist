\def\baselinestretch{1}
\chapter{Project Plan}
\ifpdf
    \graphicspath{{Theory/TheoryFigs/PNG/}{Theory/TheoryFigs/PDF/}{Theory/TheoryFigs/}}
\else
    \graphicspath{{Theory/TheoryFigs/EPS/}{Theory/TheoryFigs/}}
\fi

\def\baselinestretch{1.66}

This project is split up into three main phases. The first two phases, Analysis and Abstraction, is planned to span roughly the first two terms. The final stage, Generation, is to be completed in the third term. Some implementation details are included in this section.

\section{Analysis}

First we analyse a single poem in great depth. This means that we write algorithms to detect the use of:
\begin{itemize}
\setlength{\itemsep}{0pt}
\item{Rhyme and internal rhyme}
\item{Rhythm, including meter and syllable count}
\item{Alliteration, assonance, consonance, onomatopoeia and other sound devices}
\item{Structure, tense and repetition}
\end{itemize}

Then we try to understand the context of the poem to extract:
\begin{itemize}
\setlength{\itemsep}{0pt}
\item{Characters}
\item{Objects}
\item{Locations}
\item{Descriptions}
\item{Relationships}
\item{Actions}
\item{Point of View}
\item{Symbolism, such as metaphors and similes}
\end{itemize}

The implementations of these detectors use:
\begin{itemize}
\setlength{\itemsep}{0pt}
\item{Basic string parsing algorithms}
\item{Python Natural Language Toolkit, a suite of text processing libraries, corpora and lexical resources}
\item{The CMU Pronunciation Dictionary, which gives phonetic spellings of 133,000 English words in ARPAbet}
\item{WordNet, a lexical database that provides conceptual-semantic and lexical relations of 155,287 English words}
\item{VerbNet, a lexical database that groups verbs by semantic and syntactic linking behaviour}
\item{CLiPS Pattern, a web-mining module for Python}
\item{Written Sound, a dictionary of onomatopoeia to their meanings and associated objects}
\item{Discourse Representation Theory}
\item{Metaphor Magnet, a web application that maps commonplace metaphors in everyday texts}
\end{itemize}

This stage is currently on schedule and expected to be completed within two weeks of the submission of this paper. The tasks that remain are an improved grammar and lexicon for parsing to a Discourse Representation Structure (DRS) and metaphor detection using Metaphor Magnet.
\\\\
Note that the Theme of the poem is not be addressed at this stage. The algorithm for detecting Theme uses a very similar process to Abstraction as explained below, so has been moved to that phase.

\section{Abstraction}

A prerequisite to this stage is that all of the output from the detectors is as general as possible, leaving all options open for consideration during this phase. They should also be in an easily comparable format such as numerical scores and strings. The first task of this stage is to make sure that this is the case and make any adjustments if necessary.
\\\\
The major algorithm to be written in this stage will analyse a large set of poems, compare each aspect of the results individually and find significant overlaps. The overlapping features will be saved as a template for poems of that type. Any single type of poem may have several templates. For example, overlapping features of all limerick include its rhythm and \textit{AABBA} rhyme scheme, but only some start with \textit{"There once was a"}, so one template will be created with that start string and one without.
\\\\
We then take a large list of poems of the same Theme, as determined by classifications on poets.org [https://www.poets.org/search.php/prmResetList/1] and run it through this algorithm. We will then be able to analyse and generate poetry with Theme-associated constraints. For example, love poems may talk about roses and flowers, so we associate these objects as having a theme of love. It may also, for example, reveal that limericks are mostly humourous.
\\\\
Thereafter, we will find abstractions of the following types of poems:
\begin{itemize}
\setlength{\itemsep}{0pt}
\item{Haiku}
\item{Limerick}
\item{Sonnet}
\item{Elegy}
\item{Ode}
\item{Ballad}
\item{Cinquain}
\item{Riddle}
\item{Free Verse}
\end{itemize}

The data acquired in this stage will then be stored in a relational database.
\\\\
This algorithm can do most of its processing unsupervised and in parallel so should not take very much time. However, to be on the safe side we will employ agile methodologies to avoid running out of time for the generation phase. Firstly, we will only run abstraction on a limited number of themes. Then we will work on Haikus initially because they are short and have relatively simple structure. We will then move on to the Generation phase. Once we are able to generate Haikus, we will return to this stage and abstract the next type of poem down the list and try to generate it. We will oscillate between Abstraction and Generation in this way until the end of the list and all themes are completed, or we run out of time.

\section{Generation}

The process of generating a poem begins with a seed of inspiration. It could come in the form of a command from the user, for example "write me a limerick about a computer that is bored with data but finds poetry fun". Or it could come from another source, perhaps the top trending Tweet for that day. A type, and optionally theme, for the poem is also given, which imposes constraints on the features and form of the poem as well as preferences for the words to be used.
\\\\
The inspiration is converted into a DRS, as in the Analysis phase. As part of generating the requested type of poem, we wish to aim for the a DRS of that found in the generalisation stage for the requested poem. This enforces descriptive poems to have more descriptions or narrative ones to have more action, for example.
\\\\
The corresponding template for the requested poem is loaded. If there are multiple candidate templates, one is chosen at random. This randomness is encouraged as it could ensure non-determinism, making the poem seem more creative. An added feature would be to allow the user to alter the constraints manually and add words or phrases to be excluded from the poem. If given time, the generator could autonomously 'mutate' low priority constraints as another way of including non-determinism.
\\\\
The poem template is then filled as far as possible using the seeds only. The following algorithm is then executed to complete the rest of the poem:
\\\\
\begin{lstlisting}
For each line in the poem:
	If full sentence and valid structure, 
		continue.
	Elif full sentence and invalid structure, 
		rephrase to fit structure (using error code if provided).
	Elif almost full sentence, 
		use collocations to fill gaps.
	Elif non-empty sentence,
		determine goal of sentence based on target DRS.
		find phrases/similes/metaphors to fit.
	Elif empty sentence,
		find association to subject/object of prev line.
		add association as subject/object of sentence.
Check against constraints of type of poem.
If not all constraints are met,
	Restart loop with error codes to help rephrasing process.
\end{lstlisting}		
Words selected during the algorithm are weighted to include more alliteration and other features. A dry run example is given in the appendix. Rephrasing entails replacing a word or phrase in a sentence with a substitute that fixes the broken constraint without breaking anything further. 
\\\\
This algorithm will be implemented using all the tools listed in the Analysis section, as well as from the following sources and libraries:
\begin{itemize}
\setlength{\itemsep}{0pt}
\item{Oxford Collocations Dictionary, a source of word pairings and phrases that occur with greater than chance probability}
\item{Phrase Finder, a database of 2048 English sayings, phrases, idioms, expressions}
\item{Wordnik, a multi-faceted dictionary and semantic language database collected from a variety of sources}
\item{LinguaTools DISCO, a tool to derive semantic similarity between words based on statistical analysis of large text collections}
\item{ConceptNet 5, a semantic general knowledge network}
\item{Thesaurus Rex, a monster thesaurus generated from web texts}
\end{itemize}

Once the poem is generated, the user can once again ask for certain changes in constraints. The most useful one will perhaps be rephrasal of certain words or phrases by asking for them to be excluded. For example, if the phrase "bored to death" was used in the poem, the user could ask for the word death to be replaced. This would involve running the same algorithm as above, but with the word 'death' removed and the added constraint that the word 'death' cannot be used. This would enter the 'almost full sentence' elif condition and use collocations, possibly using 'tears' instead.

%%% ----------------------------------------------------------------------

% ------------------------------------------------------------------------

%%% Local Variables: 
%%% mode: latex
%%% TeX-master: "../thesis"
%%% End: 
