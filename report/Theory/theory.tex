\def\baselinestretch{1}
\chapter{Poem Analysis}
\ifpdf
    \graphicspath{{Theory/TheoryFigs/PNG/}{Theory/TheoryFigs/PDF/}{Theory/TheoryFigs/}}
\else
    \graphicspath{{Theory/TheoryFigs/EPS/}{Theory/TheoryFigs/}}
\fi

\def\baselinestretch{1.66}

\section{Obtaining Phonetic Structure}



\section{Rhyme}

\subsection{Detecting Rhyme}

\subsection{Rhyme Scheme Representation}



\section{Rhythm}

\subsection{Detecting Syllabic Rhythm}

\subsection{Detecting Quantitative and Accentual Rhythm}

\subsection{Rhythm Representation}



\section{Sound Devices}

\subsection{Onomatopoeia}

\subsection{Detecting Consecutive Sounds}

\subsection{Representation of Sound Device Analysis}



\section{Form and Structure}

\subsection{Stanzas, Lines and Sentences}

\subsection{Point of View}

\subsection{Tense and Aspect}



\section{Context and Pragmatics}
Here we describe the approach to the difficult challenge of determining the context of the poem in terms of its characters and their relations, descriptions etc. as described in section \ref{sec:pragpers}. The aim is to build a representation of the characters much like a human reader would in their mind. This will then be compared in the abstraction section to find a correlation between these structures and types of poetry.

\subsection{ConceptNet Relations}
ConceptNet is a semantic network of '\textit{general knowledge}'. Each node in the network is a natural language word or phrase called a \textit{concept}. Edges in the network represent a relationship between two concepts. These relationships come in various types, for example:

**Big list of the relations with the ones that we are concerned with at the top.**

We can use these relationships to build a cohesive story about particular characters during the generation phase. However, in keeping with the theme of this paper we cannot simply hard-code or randomise this process. Therefore, we need to be able to analyse these relationships in existing poems during the abstraction phase to find correlations. For example, we may find that descriptive poems have a high number of \textit{HasProperty} relations and very few characters.  

To do this, we need to be able to extract relationships between concepts in a particular poem in lieu of ConceptNet relations. For example:

**Give the cat example**

\subsection{Semantic Labelling using Noah's ARK}

It would be very difficult to determine these relationships from a syntactical parse alone. This is due to the complex nature of the English language, in particular verb usage. For example, the phrase \textit{tasted so nice} is a description rather than an action because the word \textit{tasted} in this case is being used as a linking verb, where it is usually an action verb.

These complexities are further compounded by the fact that we cannot rely on correct grammar in poems. We therefore need a \textit{semantic} parse. Noah's ARK is an informal research group run by Noah Smith at Carnegie Mellon University. They provide online API access to two tools for linguistic structure analysis, \textit{SEMAFOR} and \textit{TurboParser}. Both of these tools in conjunction are satisfactory for our aim to extract ConceptNet relations from natural language.

\subsubsection{FrameNet Semantic Role Labelling using SEMAFOR}

Semantic Role Labelling (SRL) is best described with an example. Take the sentence \textit{"The shopkeeper told the hoodlum to go away"}, we wish to recognise the verb \textit{'to tell'} as the target, \textit{'the shopkeeper'} as the speaker, \textit{'go away'} as the message and \textit{'the hoodlum'} as the addressee. This can be seen clearly in Figure BLAH. 

This is independent of the syntactic structure of the sentence and does require grammatical correctness. The SRL for the \textit{'Yoda-speak'} equivalent will remain the same, as shown in Figure BLAH.

These labels are retrieved through the SEMAFOR tool, which uses FrameNet to determine the frame-semantic structure of the text. FrameNet is a lexical database of \textit{semantic frames}, which describe the meaning of a word based on the words that typically participate with it, known as frame elements. It is based on the Frame Semantics Theory, introduced by Charles J.Fillmore et al.

We can use these to derive the ConceptNet relations by looking for the occurrence of frames, and elements thereof, that correspond to the ConceptNet relation. These manually chosen lists can be found in THE APPENDIX. However, each list may not be exhaustive for its corresponding ConceptNet relation and there are some relations that will not be picked up by this method.


\subsubsection{Semantic Dependency Relations using TurboParser}

The Stanford Dependencies is another representation based around the relationships between words. All dependency relations are strictly binary and come in various types depending on the participants (called the \textit{governor} and the \textit{dependent}).

The semantic dependency tree for the previous shopkeeper example can be seen in Figure BLAH.

This representation fills the gaps left by the frame-semantic parse using the following heuristics:

PSEUDO-CODE HEURISTICS HERE

Together, these tools provide a fairly complete coverage of the ConceptNet relations.


\subsection{Binding Relations to Characters}

At this point, the derived ConceptNet relations are only a set of abstracted FrameNet frames and matched dependencies. This alone does not give us much more information than if we were to use the frame-semantic parse on its own. The true usefulness of this approach only arises if the relations can be bound to characters of the poem. For the shopkeeper example, we would recognise \textit{the shopkeeper} and \textit{the hoodlum} as objects with SendMessage and ReceiveMessage relations bound to each of them respectively with respect to the \textit{go away} message.

We can represent the desired output as a set of 'hubs', with each character at the centre of the hub as in Figure BLAH. This will also improve our knowledge for the generation phase in that we will know the \textit{number} of characters typical for a poem type, the \textit{number and type} of relations associated with \textit{each} character, as well as allow us to find commonalities in the types of characters themselves.

To reach the desired representation, we execute the following algorithm:

for each sentence in the poem:
	get the dependency relations and frame semantic parse
	collapse loose leaves of the dependency relations
	find and create possible characters
	find candidate ConceptNet relations from the frame-semantic parse
	for each character:
		get all associated dependency relations
		for each associated dependency relation:
			if the dependent is involved in a candidate relation from the frame-semantic parse:
				bind the relation to the current character and continue
			else:
				use the heuristics for dependency relations to find possible ConceptNet relations
				bind any that are found to the current character and continue
				
Here we describe each step in detail.

\subsubsection{Obtaining the Semantic Dependency Relations and Frame-Semantic Parse from Noah's ARK}

\subsubsection{Collapsing Loose Leaves of the Semantic Dependency Relations}

\subsubsection{Finding and Creating Characters}

\subsubsection{Extracting Candidate ConceptNet Relations from the Frame-Semantic Parse}

\subsubsection{Obtaining the Associated Dependency Relations for a Particular Character}

\subsubsection{Using Semantic Dependency Relations to Extract ConceptNet Relations}

\subsubsection{Binding Extracted ConceptNet Relations Characters}
	
	

\subsection{Anaphora Resolution}



\subsection{Peeks at Future Development of Pragmatic Language Analysis}



\section{Symbolism and Imagery}

\subsection{Simile}

\subsection{Personification}


%%% ----------------------------------------------------------------------

% ------------------------------------------------------------------------

%%% Local Variables: 
%%% mode: latex
%%% TeX-master: "../thesis"
%%% End: 
