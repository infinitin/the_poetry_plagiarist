%%% Thesis Abstract --------------------------------------------------
\chapter{Abstract}
\ifpdf
    \graphicspath{{Abstract/AbstractFigs/PNG/}{Abstract/AbstractFigs/PDF/}{Abstract/AbstractFigs/}}
\else
    \graphicspath{{Abstract/AbstractFigs/EPS/}{Abstract/AbstractFigs/}}
\fi

%Motivation
Poetry is a linguistic art form designed to convey a particular, well-defined message in a memorable and powerful way. Writing poetry demonstrates precise control over syntax, comprehensive understanding of semantics and keen contextual awareness; all major challenges in computational linguistics. Poets compose works that are succinct in length but dense in meaning using symbolism. They are made pleasant to hear and easily memorable with the use of mnemonic techniques such as rhyme and rhythm. There are many types of poem, each with their own rules for usage of such techniques that have been defined and passed on over centuries without validation. Even previous attempts at automatic poetry generation have blindly hard-coded these rules, limiting the array of poems produced only to those studied by the programmer.

%Implementation
This thesis presents a full-spectrum implementation for automatic poetry composition. The first of the three phases uses third-party tools and libraries to perform an in-depth analysis of any one poem, extracting information on the usage of poetic techniques, word selection and form. We further attempt to build a semantic representation that maps out concepts mentioned in the poem and how they relate to each other. We generalise the findings over large collections of poetry, potentially of any type. This produces a blueprint that is used to guide the structure and content of the poetry generation process for that type of poem.

%Further implementation
In addition, we endeavour to build an extensive source of common-sense knowledge that will be used for inspiration and to ensure a cohesive topic flow. It closely resembles the structure of the aforementioned semantic representation built during analysis. We couple this with third-party tools that provide control over grammatical structure, enabling us to take a content-first approach to the composition process. This helps us to retain the original message of the poem while we rephrase specifically to fit the required poetic structure.

%Evaluation+Results
Our approach is evaluated in two ways; first by comparing our derived blueprints for the structure and content of popular types of poem with documented literary theory, and secondly by using a variety of quantitative and qualitative methods to asses the quality of poems produced. The results show strong correlations between our blueprint and theory. Among the exceptions are cases where our blueprint provides significant evidence against widely taught theory, such as the strict adherence to iambic pentameter in Shakespearean sonnets. The poems generated by this system have shed light on the appropriateness and effectiveness of a knowledgeable, content-first approach to poetry generation that encourages further research. 

% ----------------------------------------------------------------------


%%% Local Variables: 
%%% mode: latex
%%% TeX-master: "../thesis"
%%% End: 
