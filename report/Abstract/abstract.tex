%%% Thesis Abstract --------------------------------------------------
\chapter{Abstract}
\ifpdf
    \graphicspath{{Abstract/AbstractFigs/PNG/}{Abstract/AbstractFigs/PDF/}{Abstract/AbstractFigs/}}
\else
    \graphicspath{{Abstract/AbstractFigs/EPS/}{Abstract/AbstractFigs/}}
\fi

%Motivation
Poetry is a popular linguistic art form whose works are succinct in length but dense in symbolism and meaning. While humans have been creating and studying poetry for millennia, computers have only recently attained the necessary syntactical control, semantic understanding and contextual awareness thanks to advances in computational linguistics. 

Poems are usually of a particular type, defined by its form and use of mnemonic techniques such as rhyme and rhythm. These techniques have been passed on for generations without validation and previous attempts at automatic poetry generation have blindly hard-coded these rules, limiting the array of poems produced only to those studied by the programmer.

%Implementation
This thesis presents a full-spectrum implementation for automatic poetry analysis and composition. For any one poem, we extract form and usage of poetic techniques while attempting to build a semantic representation of how concepts mentioned in the poem relate to each other. We generalise the findings over large collections of each type, outlining a blueprint to the structure and content for generating that type of poem.

%Further implementation
In addition, we endeavour to build an extensive source of common-sense knowledge to use as inspiration and ensure a cohesive topic flow. We couple this with third-party tools that provide control over grammatical structure, enabling us to take a content-first approach to the composition process. This helps retain the original message of the poem while we rephrase specifically to fit the required poetic structure.

%Evaluation+Results
Results show strong correlations between our derived blueprint and documented theory. Among the exceptions are cases where our blueprint provides significant evidence against theory, such as the strict adherence to iambic pentameter in Shakespearean sonnets. The poems generated by this system have shed light on the appropriateness and effectiveness of a knowledge-based, content-first approach to poetry generation that encourages further investigation. 

% ----------------------------------------------------------------------


%%% Local Variables: 
%%% mode: latex
%%% TeX-master: "../thesis"
%%% End: 
